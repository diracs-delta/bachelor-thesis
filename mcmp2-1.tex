\chapter{The MC-MP2 Method I: Theory}

In this chapter, the theoretical basis of the MC-MP2 method is discussed.
Section \ref{s:mc-int} details Monte Carlo integration, and introduces the
reader to a numerical method of evaluating high-dimensional integrals, and its
relevance to the many-body electronic problem. Section \ref{s:laplace} examines
the Laplace transform, and applies it to the MP2 energy expression to obtain a
single high-dimensional integral that is suitable for Monte Carlo integration.
Finally, Section \ref{s:elec-imp} introduces a well-chosen importance function
that is necessary to evaluate the high-dimensional integral derived in the
preceding section.

\section{Monte Carlo Integration}
\label{s:mc-int}

Suppose one wishes to evaluate a multi-dimensional integral on some
\emph{finite} interval $L$,

\begin{equation}
I = \int f(\bm x) \dif \bm x,
\label{eq:mc-int}
\end{equation}

\noindent where $\bm x$ is high in dimensionality (e.g.\ 10-dimensional).
Quadrature rules become an increasingly poor choice as the dimensionality of the
system increases, as the number of integrand evaluations scales with respect to
$O(n^d)$, where $n$ is the number of points selected for the quadrature rule,
and $d$ is the dimensionality of the system. In this case, even a primitive
5-point quadrature rule, insufficient to provide accuracy for anything but
simple functions on a small interval, requires $5^{10} = 9,765,625$ integrand
evaluations.

Thus, one seeks a method of integration that converges independently of the
number of dimensions involved in the problem. This is of particular relevance to
the many-body electronic structure problem; the MP2 energy correction in
equation \ref{eq:mp2-e} requires evaluation of two 6-dimensional integrals.

One such method is \emph{Monte Carlo integration} (MC integration), which
involves simply computing an average of the integrand of randomly selected
points on $L$. The \emph{estimator}, $\hat I$, thus satisfies the property

\begin{equation}
	I \approx \hat I(N) = \frac{1}{N} \sum_{i = 1}^N f(^i \bm x),
	\label{eq:mc-method}
\end{equation}

\noindent where $^i \bm x$ is the $n$th randomly chosen value of $\bm x$ on the
interval $L$. Because the number of integrand evaluations $N$ can be chosen
prior to MC integration, this method is independent of the dimensionality of the
problem.  Given the same number of iterations, the error in MC integration
converges more rapidly than any other quadrature rule for high-dimensional
problems.

Observe that equation \ref{eq:mc-method} also satisfies the unique property that
it is \emph{massively-parallelizable}, i.e.\ a multi-threaded implementation of
the MC method is highly efficient and almost trivial. Each CPU thread only need
be supplied the function and a random input to generate a single estimator of
the integral, without any theoretical need for interprocess communication.
Afterwards, the estimators generated across all the threads are merely averaged,
and the integral evaluated using the parallel MC method is obtained. This is a
massive advantage in comparison with methods requiring linear algebra (i.e.\ HF,
CI, CCSD(T)) that are difficult to be parallelized efficiently due to the need
for interprocess communication of large arrays. For these methods, there is a
diminishing return in speed as one supplies more cores to the parallelized
implementation.

Furthermore, because $\bm x$ is treated as a random variable during MC
integration, the \emph{standard error of the integral} converges with respect to
$1/N$, and obeys the relation

\begin{equation}
	^Ns = \sqrt{\frac{\sigma^2}{N - 1}},
	\label{eq:stderr}
\end{equation}

\noindent where $\sigma^2$ can be obtained from its definition,

\begin{equation}
	\sigma^2 = \text E[(X - \mu)^2] = \int (f(\bm x) - I)^2  \dif \bm x.
\end{equation}

In exchange for a massively-parallelizable integration scheme, the main drawback
of the MC integration is the slow rate of convergence, as exhibited by equation
\ref{eq:stderr}. To reduce the uncertainty by a factor of 10, the number of
iterations must be increased by a factor of 100. If the uncertainty is already
high due to the nature of the integrand or the chosen weight function (discussed
later), then MC integration may not exhibit sufficient convergence in any
reasonable amount of time. Such practical concerns regarding variance reduction
are addressed in the next chapter.

For now, let us focus on deriving a method with which to apply MC integration to
the many-body electronic problem, specifically the calculation of the
second-order MP2 energy expression in equation \ref{eq:mp2-e}. The next step is
to determine how one should proceed with evaluating an integral over an
\emph{infinite} interval.

\subsection{Importance Sampling}

Clearly, one cannot sample an infinite interval in a finite amount of time.
Thus, a method of accurately sampling points to evaluate an integral over an
infinite interval is needed. To do so, one begins by introducing a well-chosen
\emph{weight function} $\omega(\bm x)$, such that

\begin{equation}
	I = \int \frac{f(\bm x)}{\omega(\bm x)} \omega(\bm x) \dif \bm x
	= \text E\left[\frac{f(\bm x)}{\omega(\bm x)} \right]_{\bm x \sim \omega(\bm x)},
\end{equation}

\noindent where the notation to the right of the bracket implies $\bm x$ is
sampled according to the weight function $\omega(\bm x)$. Given this new
function, the estimator then satisfies the relation

\begin{equation}
	\hat I(N) = \frac{1}{N} \sum_{i = 1}^N \frac{f(^i \bm x)}{\omega(^i \bm x)}
	\label{eq:mc-weight}
\end{equation}

\noindent where $^i\bm x$ is similarly sampled according to the weight function
$\omega(\bm x)$. This weighted estimator in equation \ref{eq:mc-weight} is
guaranteed to converge to the original integral of interest in equation
\ref{eq:mc-int}, provided the weight function satisfies the following properties.
\cite{mc-book}

\begin{itemize}
	\item $\omega(\bm x)$ has a normalized probability density function (PDF)
		which satisfies the property
		\begin{equation} \int \omega(\bm x) \dif \bm x = 1. \end{equation}

	\item  $\omega(\bm x)$ is always greater than 0, and thus does not alter
		the sign of the integral.

	\item $\omega(\bm x)$ has a slower rate of decay than $f(\bm x)$
		as it vanishes for large magnitudes of $\bm x$.

	\item $\omega(\bm x)$ satisfies the property
		\begin{equation} \left|\frac{f(\bm x)}{\omega(\bm x)}\right| < \infty \end{equation}
		except at a countable, finite number of points. In other words,
		the weighted integral should not diverge.
\end{itemize}

Another property $\omega(\bm x)$ must satisfy, specific to the evaluation of the
integrals in the MP2 energy expression, is that it must share the same
singularities as $f(\bm x)$. In other words, it should diverge at points where
$f(\bm x)$ diverges. This is important because in the expression for
$\perturbE[]{2}$, the two-electron integrals are undefined if $r_{12}$ or
$r_{34}$ are zero, and thus the weight function should not sample from such
points.

The weight function not only allows one to sample an infinite interval by using
a weighted PDF that decays as $|\bm x| \rightarrow \infty$, but also lessens
uncertainty in the estimate of $I$. Observe that the variance of
equation \ref{eq:mc-weight} is given by the relation

\begin{equation}
	\sigma^2 = \int \left( \frac{f(\bm x)}{\omega(\bm x)} - I \right)^2 \omega(\bm x) \dif \bm x
	= \int \frac{f(\bm x)^2}{\omega(\bm x)} \dif\bm x - I^2,
\end{equation}

\noindent and that a well-chosen weight function that satisfies the expression

\begin{equation}
	\omega(\bm x) \approx \frac{f(\bm x)}{I}
\end{equation}

\noindent minimizes the variance. However, since $I$ is unknown, it is best to
search for a weight function that behaves like $f(\bm x)$ everywhere, but is
easier to evaluate.

\section{Laplace Transform Formalism}
\label{s:laplace}

MC integration thus seems highly applicable to the many-body electronic problem,
and the evaluation of the 6-dimensional two-electron integrals of the MBPT
energy corrections. However, the MP2 energy expression as written in equation
\ref{eq:mp2-e} cannot be directly evaluated because of the denominator. This
section will investigate the problem of separability, the additional
transformations needed to bring about an MC-integrable MP2 energy expression,
originally forwarded by Willow and colleagues in the Hirata group \cite{willow1,
willow2}.

\subsection{Separability of the MP2 Energy}

It is unfortunate that the modern convention for denoting indices over the
occupied and virtual MOs is different from the one used by Szabo and Ostlund.
For convenience, the MP2 energy is reprinted below with the different indices.

\begin{equation}
\begin{split}
	\perturbE[]{2} &=
	\sum_{i,j}^{\text{occ}} \sum_{a,b}^{\text{vir}}
	\frac{2\braket{ij|ab}\braket{ab|ij}}{\eps_i + \eps_j - \eps_a -\eps_b}
	-
	\sum_{a,b}^{\text{occ}} \sum_{r,s}^{\text{vir}}
	\frac{\braket{ij|ab}\braket{ab|ji}}{\eps_i + \eps_j - \eps_a -\eps_b}
	\\
	&= \perturbE[\text{direct}]{2} + \perturbE[\text{exchange}]{2},
	\label{eq:mp2-ee}
\end{split}
\end{equation}

\noindent where $i,j$ and $a,b$ sum over all occupied and virtual orbitals,
respectively. MC integration, because it is an inherently stochastic method that
introduces uncertainty, is not suitable for evaluating sums of many integrals,
as the error becomes especially tedious and intractable, and the estimate of the
MP2 energy would be lost. The objective is to somehow represent this summation
into a single, 12-dimensional integral that can be evaluated by MC integration.

One can attempt to do this by multiplying the two-electron integrals together,
and expand the direct and exchange energies, but the derivation fails at the
following step.

\begin{equation}
\begin{split}
\perturbE[\text{direct}]{2}
= 2 \sum_{i,j}^{\text{occ}} \sum_{a,b}^{\text{vir}}
   &\frac{1}{\eps_i + \eps_j - \eps_a - \eps_b} \\
   &\times \left[ \iiiint \phi_i^*(\bm r_1) \phi_i(\bm r_3) \phi_j^*(\bm r_2) \phi_j(\bm r_4) \phi_a^*(\bm r_3) \phi_a(\bm r_1) \right. \\
   &\qquad\qquad\;\; \times \left. \phi_b^*(\bm r_4) \phi_b(\bm r_2) \times \frac{1}{r_{12}r_{34}} \dif \bm r_1 \dif \bm r_2 \dif \bm r_3 \dif \bm r_4 \right]
\end{split}
\end{equation}

\begin{equation}
\begin{split}
\perturbE[\text{exchange}]{2}
= - \sum_{i,j}^{\text{occ}} \sum_{a,b}^{\text{vir}}
   &\frac{1}{\eps_i + \eps_j - \eps_a - \eps_b} \\
   &\times \left[ \iiiint \phi_i^*(\bm r_1) \phi_i(\bm r_3) \phi_j^*(\bm r_2) \phi_j(\bm r_4) \phi_a^*(\bm r_4) \phi_a(\bm r_1) \right. \\
   &\qquad\qquad\;\; \times \left. \phi_b^*(\bm r_3) \phi_b(\bm r_2) \times \frac{1}{r_{12}r_{34}} \dif \bm r_1 \dif \bm r_2 \dif \bm r_3 \dif \bm r_4 \right]
\end{split}
\end{equation}

\noindent The problem is that the summations over occupied and virtual orbitals
is \emph{coupled} by the external denominator, and the sum of integrals cannot
be represented as an integral of sums (i.e.\ the summation operators cannot be
\emph{interchanged} with the 12-dimensional integral operator). Thus, some
transformation of the denominator is required to separate the denominator.

\subsection{The Laplace Transform}

Thankfully, such a mathematical transformation exists. The \emph{Laplace
transform} is defined as the transformation

\begin{equation}
	\mathcal{L}\{ f(\tau) \}(s) \equiv F(s) = \int_0^{\infty} f(t) e^{-st} \dif \tau,
\end{equation}

\noindent where $\tau$ is typically called \emph{imaginary time} in physics and
chemistry. This is especially relevant in solving differential equations as it
can transform derivatives and integrals into algebraic expressions
\cite{laplace}. However, its most relevant property for this discussion is its
action on denominators. Observe that

\begin{equation}
	\mathcal{L}\{ 1 \}\left( a + b \right) = \int_0^{\infty}e^{-(a + b)\tau} \dif \tau = \frac{1}{a + b}.
\end{equation}

\subsection{MP2 Energy in the Laplace Transform}

This observation was made by Alml{\"o}f, who derived what is now called the
\emph{Laplace transform formalism} of MP2 theory \cite{almlof1, almlof2,
almlof3}. Application of the Laplace transform to the energy denominator
in equation \ref{eq:mp2-ee} results in the transformation

\begin{equation}
\begin{split}
\mathcal{L}\{1\}(\eps_i + \eps_j - \eps_a - \eps_b) &= \int_0^\infty e^{-(\eps_i + \eps_j - \eps_a - \eps_b)\tau} \dif \tau \\
						    &= - \int_0^\infty e^{\eps_i\tau}e^{\eps_j\tau}e^{-\eps_a\tau}e^{-\eps_b\tau} \dif \tau.
\end{split}
\end{equation}

\noindent Separation of energy terms is thus achieved through such a
transformation. Upon multiplying the exponential energy terms into the integrand
of \ref{eq:mp2-ee}, the terms are finally separable, and the summation can be
moved within the integral to yield

\begin{multline}
\perturbE[\text{direct}]{2}
=
-2
\int_0^\infty
\iiiint
o(\bm r_1, \bm r_3, \tau) o(\bm r_2, \bm r_4, \tau) v(\bm r_1, \bm r_3) v(\bm r_2, \bm r_4, \tau)
\\
\times \frac{1}{r_{12}r_{34}}
\dif\bm r_1 \dif\bm r_2 \dif\bm r_3 \dif\bm r_4 \dif\tau,
\label{eq:mcmp2-dir}
\end{multline}

\noindent and

\begin{multline}
\perturbE[\text{exchange}]{2}
=
\int_0^\infty
\iiiint
o(\bm r_1, \bm r_3, \tau) o(\bm r_2, \bm r_4, \tau) v(\bm r_1, \bm r_4) v(\bm r_2, \bm r_3, \tau)
\\
\times \frac{1}{r_{12}r_{34}}
\dif\bm r_1 \dif\bm r_2 \dif\bm r_3 \dif\bm r_4 \dif\tau,
\label{eq:mcmp2-exch}
\end{multline}

\noindent where $o, v$ are defined as

\begin{align}
o(\bm r_1, \bm r_2)
&=
\sum_i^\text{occ}
\phi_i^*(\bm r_1) \phi_i(\bm r_2) e^{\eps_k \tau},
\\
v(\bm r_1, \bm r_2)
&=
\sum_a^\text{vir}
\phi_a^*(\bm r_1) \phi_a(\bm r_2) e^{- \eps_a \tau}.
\end{align}

Finally, combining the MP2 direct and exchange integrals, the MC-integrable MP2
energy is given by

\begin{equation}
\begin{split}
\perturbE[]{2}
=
\int_0^\infty
\iiiint
[
&o(\bm r_1, \bm r_3, \tau) o(\bm r_2, \bm r_4, \tau) v(\bm r_1, \bm r_4) v(\bm r_2, \bm r_3, \tau)
\\
& - 2 o(\bm r_1, \bm r_3, \tau) o(\bm r_2, \bm r_4, \tau) v(\bm r_1, \bm r_3) v(\bm r_2, \bm r_4, \tau)
]
\\
& \times \frac{1}{r_{12}r_{34}}
\dif\bm r_1 \dif\bm r_2 \dif\bm r_3 \dif\bm r_4 \dif\tau,
\label{eq:mcmp2}
\end{split}
\end{equation}

\noindent Thus, at the cost of another integral over imaginary time, one has obtained a
13-dimensional integral that can be integrated through MC integration.  However,
in practice, only the inner 12-dimensional integral over spatial coordinates
should be evaluated using the MC method. The reasoning is twofold:

\begin{enumerate}

\item The exponential term is a well-behaved, monotonically decreasing function.
	Such functions can be evaluated with quadrature rules with high accuracy
	given few more than 10--20 points. MC integration could require millions
	of iterations to achieve the same accuracy.

\item One would need to construct a new weight function to account for the
	presence of the exponential term, if imaginary time coordinates are to
	be integrated over via the MC method as well. This unnecessarily
	complicates evaluation of the integrand and slows convergence.

\end{enumerate}

\section{The Electronic Importance Function}
\label{s:elec-imp}

Because the integral is over an infinite measure, an importance function must be
specified for the inner 12-dimensional integral over spatial coordinates in
equation \ref{eq:mcmp2} before the MC method can be applied. The development of
such an importance function was forwarded by Willow and colleagues in the Hirata
group \cite{willow1, willow2}. The \emph{electronic importance function} is

\begin{equation}
\omega(\bm r_1, \bm r_2, \bm r_3, \bm r_4)
=
\frac{1}{N_g^2} \frac{g(\bm r_1) g(\bm r_2) g(\bm r_3) g(\bm r_4)}{r_{12}r_{34}}
,
\label{eq:imp}
\end{equation}

\noindent where $N_g$ is the normalization constant

\begin{equation}
N_g
=
\iint \frac{g(\bm r_1) g(\bm r_2)}{r_{12}} \dif\bm r_1 \dif\bm r_2,
\end{equation}

\noindent and $g(\bm r)$ is the sum of $s$-type \emph{Gaussian atomic orbitals}
(GTOs), each of which are centered on the atoms of system. While this is
certainly not the only possible importance function, there are several reasons
for why this is a good choice.

\begin{itemize}

\item The importance function is computationally efficient. It can be shown that
	any product of linear combination of Gaussian functions is itself a
	linear combination of Gaussian functions \cite{gaussian}. This is the
	\emph{Gaussian product theorem}, and greatly simplifies evaluation of
	the importance function.

\item Related to the preceding point, integrals of linear combinations of GTOs
	have been extensively studied for the past several decades, as Gaussian
	functions are common basis AOs in computational chemistry. Thus,
	calculation of the normalization factor is nearly trivial, for it can be
	obtained analytically \cite{gaussian-eval}.

\item The denominator cancels out the singularities of the MP2 energy, namely
	the $r_{12}r_{34}$ term.

\item Using the chosen importance function has been verified to give more rapid
	convergence than using the electron density spatial distribution as the
	importance function in existing studies \cite{willow1}.

\end{itemize}

Finally, one has obtained all the necessary components to evaluate an estimator
for the MP2 energy, obtained via the MC method. Let $f_\text{MP2}$ be the
13-dimensional integrand $\tilde f$ in equation \ref{eq:mcmp2} integrated over the
imaginary time coordinate, e.g.

\begin{equation}
f_\text{MP2}(\bm r_1, \bm r_2, \bm r_3, \bm r_4)
=
\int_0^\infty
\tilde f(\bm r_1, \bm r_2, \bm r_3, \bm r_4, \tau)
\dif\tau.
\end{equation}

\noindent Then, given the importance function in \ref{eq:imp}, the MP2 energy is
evaluated via the MC estimator

\begin{equation}
\perturbE[]{2}
\approx
\widehat{\perturbE[]{2}}
\equiv
\frac{1}{N}
\sum_{n = 1}^N
\frac
{f_\text{MP2}(^n\bm r_1, ^n\bm r_2, ^n\bm r_3, ^n\bm r_4)}
{\omega(^n\bm r_1, ^n\bm r_2, ^n\bm r_3, ^n\bm r_4)},
\label{eq:walkers}
\end{equation}

\noindent where the $n$ superscript preceding each electron coordinate denotes
that it is the $n$th randomly chosen coordinate according to the importance
function. The randomly chosen coordinates are termed \emph{walkers}, for they
``walk" across 12-dimensional space according to the importance function. This
analogy is more clear in the context of Metropolis sampling (not discussed).

\section{Conclusions}

The preceding sections proposed a method of evaluating the MP2 energy expression
via Monte Carlo integration, and discussed its applicability and scalability to
large molecular systems.  The theoretical foundation laid out in this chapter
constitutes the \emph{Monte Carlo second-order MBPT} (MC-MP2) method, and is the
subject of further investigation in this thesis. The next chapter discusses
practical enhancements to the MC-MP2 method that make it competitive with
respect to alternative methods in computational chemistry.
