\chapter{Introduction}

This chapter aims to introduce the reader to the basic mathematical and
quantum-mechanical topics needed to achieve a full understanding of the topics
presented in this thesis. This chapter borrows heavily from the textbook written
by Szabo and Ostlund, and material discussed in this chapter cites it as a
reference unless otherwise noted. \cite{szabo}

Section \ref{s:math} introduces the mathematical concepts and conventions used
throughout the text. It assumes at the very least a basic knowledge of
multivariable calculus, applied linear algebra, and the quantum mechanics
encountered in an introductory semester of physical chemistry. Section
\ref{s:problem} posits the many-body electronic problem, the central subject of
computational chemistry. Section \ref{s:hf} describes the Hartree-Fock
approximation, arguably the most important first approximation to solutions of
the many-body electronic problem. Sections \ref{s:post-hf} and \ref{s:pt} then
describe post-Hartree-Fock methods, which expand upon and refine the
Hartree-Fock approximation to provide more accurate results. In particular,
perturbation theory is discussed in greater detail due to its relevance to the
subject of this thesis, the MC-MP2 method.

\section{Mathematical Review}
\label{s:math}

\subsection{Dirac Notation}

Throughout the field of quantum mechanics, \emph{Dirac's bra-ket notation} is
widely used to express complex functions and their corresponding inner products,
and greatly lessens the burden associated with their notation. It also elegantly
encapsules the relationship between functions and vectors.

Any single-valued, continuous, square-integrable function $a(x)$ on some complex
interval can be described as a vector lying in \emph{Hilbert space}, an abstract
inner product space that extends the concept of Euclidean space to an infinite
number of dimensions. The intuition behind this one-to-one correspondence
between certain functions and infinite-dimensional vectors is best left to its
own section (see Appendix TBD).

$a(x)$ is represented as a \emph{ket} vector living in a Hilbert space termed
ket space, and is denoted as follows.

\begin{equation}
	a(x) \equiv \ket{a}
\end{equation}

Its complex conjugate, $a(x)^*$, is represented as a \emph{bra} vector living in
the dual (or, more descriptively, adjoint) space of ket space, and is denoted as

\begin{equation}
	a(x)^* \equiv \bra{a} = \ket{a}^\dagger,
\end{equation}

\noindent where the dagger denotes the adjoint. Observe that the inner product
between any two functions $a(x)$, $b(x)$ then becomes

\begin{equation}
	\int a(x)^* b(x) \dif x = \int \bra{a} \ket{b} \dif x \equiv \braket{a|b},
\end{equation}

\noindent which is exactly analogous to the definition of an inner product for
vectors.

\subsection{Atomic Units}

Atomic units arise from dropping the mathematical constants (e.g.\ $h$, $\pi$,
etc.) in each term of the \SE. For instance, the \SE\ of atomic hydrogen,

\begin{equation}
	\left(
	- \frac{\hbar^2}{2 m_e} \nabla^2
	- \frac{e^2}{4 \pi \eps_0 r}
	\right)
	\Psi = E \Psi,
\end{equation}

\noindent reduces to the form

\begin{equation}
	\left(
	- \frac{1}{2} \nabla^2
	- \frac{1}{r}
	\right)
	\Psi = E \Psi
\end{equation}

\noindent in atomic units. A brief summary of atomic units and their conversion
factors has been tabulated in Table \ref{t:atomic} for the reader's convenience.
All subsequent equations hereafter will be derived in terms of these atomic
units.

\begin{table}[]
\begin{tabular}{@{}lll@{}}
\toprule
Physical quantity & Atomic unit               & Value in SI units                                           \\ \midrule
Length            & Bohr radius               & $a_0 = 5.291772 \times 10^{-11}$ m                       \\
Mass              & Electron mass             & $m_e = 9.109383 \times 10^{-31}$ kg                      \\
Charge            & Electron charge           & $e = 1.602176 \times 10^{-19}$ C                         \\
Energy            & Hartree                   & $\frac{\hbar^2}{m_e a_0^2} = 4.359744 \times 10^{-18}$ J \\
Momentum          & Reduced Planck's constant & $\hbar = 1.054571817 \times 10^{-34}$ J $\cdot$ s           \\ \bottomrule
\end{tabular}
\caption{Table of atomic units and their conversion to SI units.}
\label{t:atomic}
\end{table}


\subsection{Electronic Orbitals}

This thesis will be primarily concerned with \emph{electronic} wavefunctions
describing the dynamics of the electrons in a molecular system, rather than
considering the total wavefunction describing the nuclei. The reasoning for
doing so is outlined in Section \ref{s:problem}.

The electronic wavefunction for a single electron is defined as a \emph{spatial
orbital}, and is denoted by $\psi(\bm r)$, where $\bm r$ represents a vector
containing the $x, y, z$ spatial coordinates of the electron. The spatial
orbital fully defines the spatial distribution of a single electron. We
furthermore define the \emph{spin orbital} $\chi(\bm x)$ as

\begin{equation}
	\chi_{j}(\bm x) =
	\begin{cases}
		\psi_{i}(\bm r) \alpha(\omega)  & \text{if } j = 2i - 1, \\
		\psi_{i}(\bm r) \beta(\omega)   & \text{if } j = 2i, \\
	\end{cases}
	\label{eq:spino}
\end{equation}

\noindent where $\bm x$ is a four-vector containing the three spatial
coordinates (i.e.\ the components of $\bm r$) and the additional spin coordinate
$\omega$. Here, $\alpha(\omega), \beta(\omega)$ are any two arbitrary orthogonal
functions of some arbitrary spin coordinate $\omega$, and represent the electron
having $m_s = \pm \frac{1}{2}$ respectively. Thus, the spin orbital also
distinguishes between electrons that may have the same spatial distribution, but
have different spins. An example of this is the singlet ground state of
molecular hydrogen, where the electrons occupy the same spatial 1s molecular
orbital (MO), but have opposing spins.  Thus, the two electrons are said to
occupy separate spin orbitals.

It is now worth mentioning the conventions for indexing the spatial and spin
orbitals. Typically, the number of electrons in a system is represented by $N$,
while the total number of possible spatial orbitals (not just the occupied ones)
is given by $K$, where $K \geq \frac{N}{2}$. Thus, the total number of spin orbitals is
given by $2K$, directly following from equation \ref{eq:spino}. This convention
will be followed throughout the rest of this thesis.

\subsection{Hartree Products}

The full Hamiltonian of a molecular system includes two-body interactions that
complicate derivation of the solutions. However, if the Hamiltonian can be
approximated as a series of one-electron interactions, e.g.

\begin{equation}
	\hamil \approx \sum_i^N \hat h(\bm r_i) ,
	\label{eq:onehamil}
\end{equation}

\noindent where $\hat h(\bm r_i)$ is a one-electron Hamiltonian that
\emph{approximates} the contribution of the $i$th electron to the total energy,
e.g.\ by somehow averaging the contribution of other electrons out (c.f. Section
\ref{s:hf}).

Given this approximation, if the series of one-electron Hamiltonians is solved
for all $i$, and the spin orbitals $\{ \chi_j \}$ are known to satisfy the
relation

\begin{equation}
	\hat h(\bm r_i) \chi_j (\bm x_i) = \epsilon_j \chi_j(\bm x_i),
\end{equation}

\noindent then it can be shown that the corresponding \emph{approximate} total
electronic wavefunction $\Psi$, the eigenfunction satisfying equation
\ref{eq:onehamil}, is given by the \emph{Hartree product}

\begin{equation}
	\Psi(\bm x_1, \bm x_2, \dots, \bm x_N) = \chi_i(\bm x_1) \chi_j(\bm x_2) \dots \chi_k(\bm x_N),
	\label{eq:hartree}
\end{equation}

\noindent with a corresponding eigenvalue

\begin{equation}
	E = \eps_i + \eps_j + \cdots + \eps_k.
\end{equation}

Note that the Hartree product approximates any possible state of the molecular
system, not just the ground state, hence the unusual indexing of the spin
orbitals in equation \ref{eq:hartree}. $i, j, k$ need not satisfy any relation
other than that they are distinct and less than or equal to $2K$, and thus the
Hartree product can represent the either the ground state or any arbitrary
excited state. This makes Hartree products useful first approximations to the
true nature of the total electronic wavefunction.

\subsection{Slater Determinants}

However, even forgiving the fact that the Hartree product is an approximation to
the total electronic wavefunction, there exists a fundamental failure of the
Hartree product in its ability to describe real quantum systems.  Namely, it
does not treat electrons as fermions, which necessarily satisfy the Pauli
exclusion principle. The Pauli exclusion principle in turn requires that all
electronic wavefunctions must be antisymmetric with respect to exchange, i.e.

\begin{multline}
	\Psi(\bm x_1, \bm x_2, \dots, \bm x_i, \dots, \bm x_j, \dots, \bm x_N) \\
	= - \Psi(\bm x_1, \bm x_2, \dots, \bm x_j, \dots, \bm x_i, \dots, \bm x_N).
\end{multline}

Because of the commutativity of simple multiplication, the Hartree product does
not satisfy this relation, and thus is not a satisfactory description of the
electronic wavefunction. Instead, there must exist a more complex relationship
between the total electronic wavefunction and the one-electron spin orbitals
satisfying equation \ref{eq:onehamil}. One method of antisymmetrizing the total
electronic wavefunction is to represent it as a normalized determinant of a
matrix of spin orbitals, where the $i$th row and $j$th column contain the spin
orbital $\chi_j(\bm x_i)$, i.e.

\begin{equation}
	\Psi(\bm x_1, \bm x_2, \dots, \bm x_N)
	=
	\frac{1}{\sqrt{N!}}
	\begin{vmatrix}
		\chi_i (\bm x_1) & \chi_j (\bm x_1) & \cdots & \chi_k (\bm x_1) \\
		\chi_i (\bm x_2) & \chi_j (\bm x_2) & \cdots & \chi_k (\bm x_2) \\
		\vdots & \vdots & \ddots & \vdots \\
		\chi_i (\bm x_N) & \chi_j (\bm x_N) & \cdots & \chi_k (\bm x_N)
	\end{vmatrix}
	.
	\label{eq:slater}
\end{equation}

Equation \ref{eq:slater} defines the \emph{Slater determinant}, and is the
antisymmetrized total electronic wavefunction of a $N$-electron system. The
antisymmetry property of the Slater determinant can be verified by noting that
the interchange of any two electrons is equivalent to interchanging two rows in
the determinant. The interchange of any two rows or columns of a determinant
reverses the sign. This is made especially apparent in the \emph{Leibniz formula
for determinants},

\begin{equation}
	|A| = \sum_{\sigma \in S_n} \text{sgn}(\sigma) \prod^n_i A_{\sigma(i), i},
\end{equation}

\noindent where $\sigma$ is an arbitrary permutation in the set of all $n!$
possible permutations $S_n$, and $\text{sgn}(\sigma)$ is the signature of the
permutation, which is equal to unity and reverses sign following each single
interchange of indices.

\section{The Many-body Electronic Problem}
\label{s:problem}

Given the mathematical review in the preceding section, one is now in a position
to discuss the many-body electronic problem and the computational difficulty
involved in its solution. The central interest of electronic structure theory is
the determination of solutions to the non-relativistic, time-independent \SE

\begin{equation}
	\hamil \ket{\Psi(\{\bm{r_i}\} , \{\bm{R_I}\}} = E \kpsi
\label{eq:tise}
\end{equation}

\noindent where $\kpsi$ is an exact wavefunction for the molecular system,
describing both the nuclei and electrons simultaneously, and $E$ is the
corresponding energy of $\kpsi$. The Hamiltonian is in turn given by the
relation

\begin{equation}
	\hamil = - \sum_i^n \frac{1}{2}\nabla_i^2
	- \sum_I^N \frac{1}{2}\nabla_I^2
	 - \sum_i^n \sum_I^N \frac{Z_I}{r_{iI}}
	 + \sum_i^n \sum_{j > i}^n \frac{1}{r_{ij}}
	 + \sum_I^N \sum_{J > I}^N \frac{Z_I Z_J}{r_{AB}}.
\label{eq:fullh}
\end{equation}

Here, lowercase indices index over the $n$ electrons, while uppercase indices
index over the $N$ nuclei. $Z_I$ represents the charge of a given nucleus $I$,
and $r$ has the usual definition of being the distance between any two pairs of
coordinates. A closer observation reveals the intuition summarized in the above
equation. The first and second terms denote the kinetic energy of the electrons
and nuclei respectively, while the third, fourth, and fifth terms denote Coulomb
attraction and repulsion within the molecular system.

As written above, equation \ref{eq:tise} is an extraordinarily involved $3(n +
N)$-dimensional non-linear partial differential equation (PDE), with no
analytical solution known for molecular systems larger than hydrogen.
Furthermore, the eigenvalues and eigenfunctions of equation \ref{eq:tise} are
seldom desired. Nuclear coordinates are typically already known via experimental
techniques such as X-ray diffraction. The computational chemist wishes to
determine purely the electronic component of the wavefunction and its
corresponding eigenvalues, both of which collectively govern pertinent chemical
properties such as polarizability, polarity, reactivity, and optical spectra.

Thus, further approximations must be made to simplify equations \ref{eq:tise}
and \ref{eq:fullh} and isolate the many-body electronic problem.

\subsection{Non-relativistic Approximation}

It is worth mentioning that the written form of \ref{eq:fullh} already
encompasses the non-relativistic approximation, which neglects special
relativistic effects electrons experience with sufficiently high linear and
angular momenta \cite{rel1, rel2}. While this effect is especially significant
for heavy atoms such as transition and post-transition metals (see the seminal
reviews of Pyykk{\"o} and coworkers \cite{pyykko1, pyykko2, pyykko3}), it
typically results in little error for organic systems such as those investigated
hereonafter, and thus can be neglected.

\subsection{Born-Oppenheimer Approximation}

Nuclei are orders of magnitude more massive than electrons, and thus experience
different time scales of motion, i.e.\ nuclei move far more slowly than
electrons given the same momentum. In the context of the many-body electronic
problem, this implies that nuclei can be essentially regarded as fixed points in
space, and can be treated as given parameters to the problem. In other words,
the problem is said to depend on the nuclear coordinates \emph{parametrically}
rather than explicitly. This does not imply that electron-nuclei interactions
neglected, but rather that the interactions are \emph{de-coupled} from one
another, and that one-body nuclear interactions treated as constants and
two-body electron-nuclei interactions are treated as one-body interactions.

Thus, the nuclear terms (the second and fifth terms of equation \ref{eq:fullh})
can be dropped and treated as added constants to the final energy of the
molecular state. These dropped nuclear terms include the translational,
vibrational, and rotational energies of the molecule. Equations \ref{eq:tise}
and \ref{eq:fullh} reduce to

\begin{equation}
	\hamil \ket{\Psi(\{\bm{r_i}\} ; \{\bm{R_I}\}} = E \kpsi,
\label{eq:BO}
\end{equation}

\noindent where $\kpsi$, $E$ now represent the \emph{electronic} wavefunction
and its corresponding energy, and

\begin{equation}
\hamil = - \sum_i^n \frac{1}{2} \nabla_i^2
	 - \sum_i^n \sum_I^N \frac{Z_I}{r_{iI}}
	 + \sum_i^n \sum_{j > i}^n \frac{1}{r_{ij}}.
\label{eq:BOhamil}
\end{equation}

\subsection{Statement of the Problem}

The many-body electronic problem is summarized in equations \ref{eq:BO} and
\ref{eq:BOhamil}, and can be made explicit as follows: \emph{given a set of
nuclear coordinates $\{ R_I \}$, what are the corresponding wavefunctions and
energies which satisfy the electronic \SE ?}

No general method has been devised to directly solve the many-body electronic
problem. Even with the drastic simplification conferred by the non-relativistic
approximation, and the $3N$ fewer degrees of freedom resulting from the
Born-Oppenheimer approximation, equation \ref{eq:BO} remains a $3n$-dimensional
non-linear PDE with no analytical solution. The subsequent sections in this
chapter detail further approximations one can make to arrive at numerical
solutions to the many-body electronic problem.

\section{The Hartree-Fock Approximation}
\label{s:hf}

\subsection{Physicists' Notation}

Prior to further discussion, it is worth mentioning a notational shorthand for
one-electron and two-electron Coulomb integrals that are so commonly encountered
in further derivation of Hartree-Fock methods. This notation is known as
\emph{physicists' notation}. For the one-electron case,

\begin{equation}
	\braket{a|h|b} = \int \chi_a^* (\bm x_i) h(\bm r_i) \chi_b (\bm x_i) \dif x_i,
\end{equation}

\noindent and for the two-electron case,

\begin{equation}
	\braket{ab|cd} = \iint \chi_a^* (\bm x_i) \chi_b^*(\bm x_j)
	\frac{1}{r_{ij}}
	\chi_c (\bm x_i) \chi_d (\bm x_j) \dif \bm x_i \dif \bm x_j.
\end{equation}

It is implied that during integration, spin orbitals lying left of the central
separator are transformed to their corresponding complex conjugates, and are
functions of $\bm x_i, \bm x_j$ respectively. Spin orbitals lying to the right
are similarly defined, but are left unaffected during integration.

One can furthermore define the \emph{anti-symmetrized} two-electron Coulomb
integration in physicists' notation through the addition of an additional
vertical separator as follows.

\begin{equation}
	\bra{ab} \ket{cd} = \braket{ab|cd} - \braket{ab|dc}.
\end{equation}

\subsection{The Hartree-Fock Equations}

The most troublesome aspect of the many-body electronic problem is the two-body
Coulomb interaction which makes it inseparable. However, to approximate the
two-body Coulomb interaction that a single electron experiences, one could
compute an average potential generated from all other electrons, thus
simplifying the many-body electronic problem.

The \emph{Hartree-Fock (HF) approximation} achieves this by decomposing the
N-electron \SE\ detailed in equations \ref{eq:BO} and \ref{eq:BOhamil} into $N$
one-electron \emph{Hartree-Fock equations}, which satisfy the relation

\begin{equation}
\begin{split}
	\label{eq:fock}
	\hat f(\bm x_i) \chi_a (\bm x_i)
	&= \hat h(\bm x_i) \chi_a (\bm x_i) \\
	&\quad + \sum_{b \neq a}^{2K} \left[ \int |\chi_b(\bm x_j)|^2 \frac{1}{r_{ij}} \dif \bm x_j \right] \chi_a (\bm x_i) \\
	&\quad - \sum_{b \neq a}^{2K} \left[ \int \chi_b^*(\bm x_j) \chi_a(\bm x_j) \frac{1}{r_{ij}} \dif \bm x_j \right] \chi_b (\bm x_i) \\
	&= \eps_a \chi_a (\bm x_i)
\end{split}
\end{equation}

\noindent for all occupied spin orbitals, where $\hat f(\bm x_i)$ is termed the
\emph{Fock operator}, $\eps_a$ is the energy of the spin orbital $\chi_a$
occupied by electron $i$, $K$ is the number of given spatial basis functions,
and $\hat h (\bm x_i)$ is the simplified one-electron Hamiltonian

\begin{equation}
\hat h (\bm x_i) = - \frac{1}{2} \nabla_i^2
	           - \sum_I^M \frac{Z_I}{r_{iI}},
\end{equation}

\noindent given $M$ nuclei.

The seemingly complicated integro-differential Hartree-Fock equations in
equation \ref{eq:fock} can be decomposed term-by-term to assess the intuition
behind the expression. The bracketed quantity in the second term can be
interpreted as a \emph{mean-field Coulomb operator}, which computes the average
potential by the $i$th electron generated by an electron occupying the $b$th
spin orbital. This can be made explicit in the expression

\begin{equation}
\hat J_b (\bm x_i) = \int |\chi_b (\bm x_j)|^2 \frac{1}{r_{ij}} \dif \bm x_j.
\end{equation}

The third term in equation \ref{eq:fock} can also be written in terms of a new
operator known as the \emph{exchange operator}, though this does not have a
simple classical interpretation as did the mean-field Coulomb operator, and
arises purely from the antisymmetric nature of the wavefunction as expressed by
the Slater determinant. The action of the exchange operator on a spin orbital is
given by the equation

\begin{equation}
	\hat K_b (\bm x_i) \chi_a(\bm x_i) =
	\left[ \int \chi_b^* (\bm x_j) \frac{1}{r_{ij}} \chi_a(\bm x_j) \dif \bm x_j \right]
	\chi_b (\bm x_i).
\end{equation}

An ad-hoc justification for the presence of this term in the Hartree-Fock
equations is to account for the lower energy of systems of parallel spins, a
quantum-mechanical phenomenon known as \emph{exchange correlation}. This process
can be rationalized by noting that electrons with parallel spins have an
inherently lower probability of being near each other because of Pauli
exclusion, and thus experience less classical Coulomb repulsion between one
another, lowering the energy of the system.

Given the Coulomb and exchange operators, the Hartree-Fock equations can be
re-written as

\begin{equation}
\hat f(\bm x_i) \chi_a (\bm x_i) =
\left[ h(\bm x_i)
+ \sum_{b \neq a}^{2K} \hat J_b (\bm x_i)
- \sum_{b \neq a}^{2K} \hat K_b (\bm x_i) \right] \chi_a (\bm x_i)
= \eps_a \chi_a(\bm x_i).
\label{eq:hf}
\end{equation}

\subsection{The Self-Consistent Field Method}

The Hartree-Fock equations expressed in equation \ref{eq:hf} still present with
immense computational difficulty, in that the Fock operator depends on the
eigenfunctions themselves. Thus, the Hartree-Fock equations are
\emph{pseudo-eigenvalue problems}, as the operator itself depends on the
solutions obtained from the operator, and can only be solved iteratively.
Nevertheless, there at least exists a clear method for solving this problem,
dubbed the \emph{self-consistent field method} (SCF method). The general
theoretical procedure is as follows.

\begin{enumerate}

	\item Assuming one has already been given a molecule (i.e.\ the set of
		nuclear coordinates, atomic numbers, and number of electrons),
		one must provide a basis set of spatial \emph{atomic orbitals}
		(AOs). Many different basis sets are available in the literature
		for varying applications, and it is to the chemists' discretion
		which to use.

	\item Generate the Fock operator using the given basis set.

	\item The generated Fock operator to solve for the new eigenvalues and
		new eigenvectors, which are closer approximations to the true
		eigenvalues and eigenvectors of the Fock operator. The new
		eigenvectors are linear combinations of the AOs, and are termed
		\emph{molecular orbitals} (MOs).

	\item Repeat steps 2 and 3 until the eigenvalues and eigenvectors reach
		sufficient convergence.

\end{enumerate}

Ideally, this provides a set of MOs defined in terms of the AO basis, with a
corresponding set of eigenvalues. Of course, this by no means provides a
practical implementation of the HF method. In practice, implementation of the HF
method is complicated by issues such as non-orthogonal basis functions, and
requires inclusion of more sophisticated techniques in numerical linear algebra
such as diagonalization and unitary transformations. This work will not discuss
such techniques, as its primary intent is to illustrate developments in MC-MP2
theory, and will treat obtained HF energies and orbitals as givens.
Nevertheless, MP2 builds off of the theoretical foundations laid out by the HF
method, and it is instructive to learn the fundamentals of the HF method as to
improve upon its results via MP2 theory.

% \subsection{The Hartree-Fock Energy}
%
% Assuming that the SCF method obtained the set of eigenvalues and eigenfunctions
% of the Fock operator, it is then sufficient to use these to obtain the
% \emph{Hartree-Fock energy}.

\section{Post-Hartree-Fock Methods}
\label{s:post-hf}

\emph{Post-Hartree-Fock methods} are computational methods aimed at achieving
more accurate solutions to the many-body electronic problem than those obtained
via the HF method. Each of these methods corrects for one (of multiple)
approximations made during solution of the HF equations. One such method,
second-order many-body perturbation theory (MP2), is the central topic of this
thesis, and is discussed in its own subsequent section. This section aims at
discussing alternative post-Hartree-Fock methods and their various shortcomings
as motivations for further development of MP2 theory.

\subsection{Configuration Interaction}

In the HF method, the ground state $N$-electron total electronic wavefunction is
evaluated as the Slater determinant of $N$ single-electron eigenfunctions of the
Fock operator. This incomplete, approximate representation of the total
electronic wavefunction necessarily over-estimates the ground state energy, per
the variational principle.

To achieve a finer upper bound to the ground state energy, one can include
contributions from other Slater determinants formed from the $N$ electrons all
$2K$ spin orbitals, not just those lowest in in energy. This requires the
computation of

\begin{equation}
	\binom{2K}{N} = \frac{2K!}{N!(2K - N)!}
\end{equation}

\noindent Slater determinants. Thus, the \emph{method of configuration
interaction (CI)} aims at solving the variational problem

\begin{equation}
	\Psi = c_o \ket{\Psi_0} +  \sum_{a} \sum_{r} c_a^r \ket{\Psi_a^r}
	+ \sum_{a} \sum_{b > a} \sum_{r} \sum_{r > s} c_{ab}^{rs} \ket{\Psi_{ab}^{rs}}
	+ \cdots,
\label{eq:full-ci}
\end{equation}

\noindent with the second term representing a summation over all singly-excited
Slater determinants, the third term representing a summation over all
doubly-excited Slater determinants, and so on. Needless to say, solution of this
variational problem for all possible Slater determinants (full CI) is
extraordinarily computationally complex as the number of Slater determinants
required grows on the order of $2K!$. A full CI calculation of just molecular
nitrogen using the ANO basis (4s3p1d) requires $9.68 \times 10^9$ Slater
determinants. \cite{n2-ci}

With full CI being prohibitively expensive, an alternative is to truncate the
above variational problem past the doubly-excited Slater determinants in
equation \ref{eq:full-ci}, in a procedure known as singly and doubly excited CI
(SDCI). Alternatively, one can choose to vary the spin orbitals within the
Slater determinant simultaneously while varying the set of coefficients to
counteract the error introduced via truncation, in a procedure known as
multiconfiguration self-consistent-field (MCSCF).

However, truncated CI also suffers from a serious failure in that it fails to be
\emph{size-consistent}, i.e. its energy fails to scale with $N$ given $N$
interacting systems. In fact, its variational estimate of the energy tends
towards zero in the limit $N \rightarrow \infty$. For investigations into
molecular dimers (such as the ones discussed in this thesis) or larger systems
such as crystals, this shortcoming is fatal.  An investigation of why this is
true in the general case of truncated CI is rather involved and well beyond the
scope of this text.  Nevertheless, a brief mention of this shortcoming, in
conjunction with the computational cost of CI methods in general, is enough to
rationalize the relative scarcity of CI methods in computational chemistry
today.

\subsection{Pair and Coupled Pair Theories}

Pair and coupled pair theories are a broad family of methods that essentially
provide size-consistent approximations to the full CI method. This includes the
independent electron pair approximation (IEPA) and the coupled cluster (CC)
method. Of these theories, the CC method is the most competitive. A specific
instance of the CC method, CCSD(T) (coupled cluster single double triple)
provides some of the most highly accurate results available in computational
chemistry. \cite{cc1, cc2, cc3, cc4}

Thus, CC methods serve as a competitive benchmark for correlation energies
obtained using the MC-MP2 method discussed in this thesis. However, its
computational cost scaling makes it prohibitively expensive for very large
systems. Thus, further development of post-Hartree-Fock methods are still needed
to analyze very large molecular systems.

\section{Perturbation Theory}
\label{s:pt}

Perturbation theory is a theory encompassing a broad range of methods in
computational physics and chemistry aimed at improving approximations to complex
systems by constructing the application of successive refinements known as to
the eigenvalues and eigenfunctions of the approximate system. Here, the two most
pertinent examples to the many-body electronic structure problem are discussed.

\subsection{Rayleigh-Schr{\"o}dinger Perturbation Theory}

Suppose one is already given the eigenfunctions and eigenvalues of an
\emph{approximate} Hamiltonian $\hamil_0$, which relates to the full Hamiltonian
$\hamil$ through the addition of \emph{perturbation} $\hat V$, i.e.

\begin{equation}
	\hamil\ket{\Psi_i} = (\hamil_0 + \lambda\hat V) \ket{\Psi_i} = E_i \ket{\Psi_i}
	\label{eq:pt}
\end{equation}

\noindent where $\lambda$ is equal to unity, and used purely to keep track of
terms for future derivation. Next, let the eigenfunctions and eigenvalues of
$\hamil$ be expanded into a power series relation with respect to its
approximated counterparts.

\begin{align}
	\label{eq:e-pt}
	E_i &= E^{(0)}_i + \lambda E^{(1)}_i + \lambda^2 E^{(2)}_i + \cdots \\
	\label{eq:psi-pt}
	\ket{\Psi_i} &= \ket{\Psi^{(0)}_i} + \lambda \ket{\Psi^{(1)}_i} + \lambda^2 \ket{\Psi^{(2)}_i} + \cdots
\end{align}

\noindent where the superscripts represent the successively finer corrections to
the energy and wavefunction. For instance, $E_i^{(2)}$ represents the
second-order correction to the $i$th energy. This expansion of the eigenvalues
and eigenfunctions into a sum of successive corrections to their zeroth-order
approximations is \emph{Rayleigh-Schr{\"o}dinger perturbation theory} (RSPT). To
simplify this discussion, the expressions for third and higher-order
perturbations will not be discussed, for their expressions are exceedingly
cumbersome, and serve little purpose in the discussion of the MC-MP2 method, a
variant of second-order perturbation theory.

Next, let equations \ref{eq:e-pt} and \ref{eq:psi-pt} be substituted into the
perturbation expansion \ref{eq:pt}. Gathering the zeroth, first, and second
order terms corresponding to $1, \lambda, \lambda^2$ respectively, one finds

\begin{align}
	\hamil_0 \perturbPsi{0} &= \perturbE{0} \perturbPsi{0} \\
	\hamil_0 \perturbPsi{1} + \hat V \perturbPsi{0}
				&= \perturbE{0} \perturbPsi{1} + \perturbE{1} \perturbPsi{0} \\
	\hamil_0 \perturbPsi{2} + \hat V \perturbPsi{1}
				&= \perturbE{0} \perturbPsi{2} + \perturbE{1} \perturbPsi{1} + \perturbE{2} \perturbPsi{0}.
\end{align}

Further simplification of the above expressions by taking advantage of the
orthogonality and completeness of the set $\{\perturbPsi{0}\}$ results in the
following expressions for the zeroth, first, and second-order energies.

\begin{align}
	\perturbE{0} &= \braket{\Psi_i^{(0)} | \hamil_0 | \Psi_i^{(0)}} \\
	\perturbE{1} &= \braket{\Psi_i^{(0)} | \hat V | \Psi_i^{(0)}} \\
	\label{eq:rspt-2}
	\perturbE{2} &= \sum_{j \neq i}
	\frac{\braket{\Psi_i^{(0)} | \hat V | \Psi_j^{(0)}} \braket{\Psi_j^{(0)} | \hat V | \Psi_i^{(0)}}}
	{\perturbE[i]{0} - \perturbE[j]{0}}.
\end{align}

\subsection{Many-body Perturbation Theory}

Application of RSPT to the many-body electronic problem is termed
\emph{many-body perturbation theory} (MBPT), and aims at improving the
approximate energies and wavefunctions obtained by the HF method. In MBPT, a
two-body perturbation, which removes the mean-field approximation and replaces
it with the two-electron Coulomb operator, is applied to the Hartree-Fock
operator. This can be explicit as follows.

\begin{align}
	\hamil_0 &=  \sum_i \hat f(\bm x_i) = \sum_i \left[ \hat h(\bm x_i) + \hat v^\text{HF}(\bm x_i) \right], \\
	\hat V &= \sum_i \sum_{j > i} \frac{1}{r_{ij}} - \sum_i \hat v^\text{HF}(\bm x_i).
\end{align}

The second-order MBPT (termed \emph{MP2}) energy expression can then be derived
using the second-order RSPT energy expression in \ref{eq:rspt-2} through
substitution into equatoin \ref{eq:pt} and simplification of two-electron
integrals. The rules for simplifying such integrals are given in Appendix TBD.
It turns out that only doubly-excited determinants contribute to the MP2
correction, and the final expression for a \emph{closed-shell system} (with no
unpaired electrons in any spatial orbital) is given by the relation

\begin{equation}
	\perturbE[]{2} =
	\sum_{a,b}^{\text{occ}} \sum_{r,s}^{\text{vir}}
	\left[
	\frac{2\braket{ab|rs}\braket{rs|ab}}{\eps_a + \eps_b - \eps_r -\eps_s}
	-
	\frac{\braket{ab|rs}\braket{rs|ba}}{\eps_a + \eps_b - \eps_r -\eps_s}
	\right],
	\label{eq:mp2-e}
\end{equation}

\noindent where $a,b$ sum over the occupied spatial orbitals and $r,s$ sum over
the unoccupied spatial orbitals, appropriately termed \emph{virtual orbitals}.
For future reference, it is useful to separate the summation in equation
\ref{eq:mp2-e} into two separate summations containing the first and second
terms, and denote them as ``direct" and ``exchange" energies respectively, i.e.

\begin{equation}
\begin{split}
	\perturbE[]{2} &=
	\sum_{a,b}^{\text{occ}} \sum_{r,s}^{\text{vir}}
	\frac{2\braket{ab|rs}\braket{rs|ab}}{\eps_a + \eps_b - \eps_r -\eps_s}
	-
	\sum_{a,b}^{\text{occ}} \sum_{r,s}^{\text{vir}}
	\frac{\braket{ab|rs}\braket{rs|ba}}{\eps_a + \eps_b - \eps_r -\eps_s}
	\\
	&= \perturbE[\text{direct}]{2} + \perturbE[\text{exchange}]{2}.
\end{split}
\end{equation}

\subsection{Computational Complexity of MP2}

Despite being a size-consistent alternative to CI methods, $k$th order MBPT
methods suffer from a high computational cost scaling similar to that of the
coupled cluster methods. The computational complexity is bounded by
$O(n^{k+3})$, where $n$ is some measure proportional to the size of molecular
system (e.g. number of basis sets) \cite{mbpt-book, mp2-direct}. Interestingly,
the bottleneck of the traditional (hereafter termed \emph{direct}) MP2 algorithm
is not the evaluation of the two-electron integrals, but rather the
transformation of the integral of basis AOs to the integral of MOs using
coefficient matrices desired in equation \ref{eq:mp2-e}, i.e.

\begin{equation}
	\braket{pq|rs} = \sum_\kappa \sum_\lambda \sum_\mu \sum_\nu
	C_p^{\kappa *} C_q^{\lambda *} C_r^\mu C_s^\nu \braket{\kappa\lambda | \mu\nu},
	\label{eq:aomo-trans}
\end{equation}

\noindent where $p,q,r,s$ denote arbitrary MOs that are linear combinations of
the basis AOs $\kappa,\lambda,\mu,\nu$ related by the transformation matrices
$C_p^{\kappa}, C_q^{\lambda}, C_r^\mu, C_s^\nu$ generated by the HF method.

Despite this transformation being costly to execute each iteration, it is
usually still more rapid than pre-computing and storing the integrals, and then
loading the integrals from memory. This is especially true for parallel
supercomputing applications, where interprocess communication is costly.
Furthermore, not many integrals are reused in MP2 energy expression, and thus
memory I/O is even more inefficient with respect to computing the transformation
in equation \ref{eq:aomo-trans} each iteration.

To understand why this may be so, it is helpful to consider the following
analogy. The scaled time difference between one CPU cycle (assuming a 3 GHz CPU)
and loading from DIMM RAM is the difference equivalent to the difference between
one second and four minutes \cite{systems}. If the number of MO integrals is too
large to be stored in DIMM RAM, the computational cost is drastically worse; the
scaled time requirement (again, given a CPU cycle is one second) of loading from
external memory is anywhere between 4 days for SSDs to 9 months for HDDs.  Thus,
in the case of MP2 methods, the AO-MO transformation in equation
\ref{eq:aomo-trans} is a painful but necessary step.

To enhance the applicability of MBPT methods, specifically MP2, more
sophisticated, parallelized implementations of MBPT methods are needed. The next
chapter will discuss the MC-MP2 method forwarded by the Hirata group, and its
relevance towards obtaining solutions to the many-body electronic problem for
very large systems.


