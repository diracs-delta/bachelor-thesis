\chapter{The MC-MP2 Method II: Implementation}

In the preceding chapter, the theoretical background of the MC-MP2 method was
discussed. This chapter focuses on the additional methodology needed for a
robust and practical implementation of the MC-MP2 method, and includes
discussion of some of my personal contributions to the MC-MP2 method code
written by Doran.

\section{Inverse Sampling}

\section{The Redundant-Walker Algorithm}

Unfortunately, direct implementation of the

\section{The Control Variates Method}

\section{Pseudorandom Number Generation}

Pseudorandom number generation is a non-trivial issue in the field of computer
science, because truly random numbers are difficult to generate, and are
typically limited by hardware I/O. A complete discussion and comparison of
random number generation methods is beyond the scope of this text. However, it
suffices to mention the general process. \emph{Pseudorandom number generators}
(PRNGs) generate a sequence of numbers that approximate sequences generated by
truly random processes. However, they are deterministic, and the sequence is
entirely dependent on the \emph{seed} provided to the PRNG. The seed value is
just a parameter chosen to generate different random sequences each iteration,
and is typically chosen from a truly random input, e.g.\ analog hardware noise.
For all subsequent experimental work, the \emph{Mersenne Twister algorithm} is
assumed.

\section{Inverse Transform Sampling}
